\PassOptionsToPackage{svgnames,dvipsnames,svgnames}{xcolor}
\documentclass[sigplan,10pt]{acmart}
\settopmatter{printfolios=true,printccs=true,printacmref=true}
%% For double-blind review submission, w/ CCS and ACM Reference
%\documentclass[acmsmall,review,anonymous]{acmart}\settopmatter{printfolios=true}
%% For single-blind review submission, w/o CCS and ACM Reference (max submission space)
%\documentclass[acmsmall,review]{acmart}\settopmatter{printfolios=true,printccs=false,printacmref=false}
%% For single-blind review submission, w/ CCS and ACM Reference
%\documentclass[acmsmall,review]{acmart}\settopmatter{printfolios=true}
%% For final camera-ready submission, w/ required CCS and ACM Reference
%\documentclass[acmsmall]{acmart}\settopmatter{}


%% Copyright information
%% Supplied to authors (based on authors' rights management selection;
%% see authors.acm.org) by publisher for camera-ready submission;
%% use 'none' for review submission.
% \setcopyright{none}
%\setcopyright{acmcopyright}
%\setcopyright{acmlicensed}
%\setcopyright{rightsretained}
%\copyrightyear{2018}           %% If different from \acmYear

%% Bibliography style
% \bibliographystyle{ACM-Reference-Format}
%% Citation style
%\citestyle{acmauthoryear}  %% For author/year citations
%\citestyle{acmnumeric}     %% For numeric citations
%\setcitestyle{nosort}      %% With 'acmnumeric', to disable automatic
                            %% sorting of references within a single citation;
                            %% e.g., \cite{Smith99,Carpenter05,Baker12}
                            %% rendered as [14,5,2] rather than [2,5,14].
%\setcitesyle{nocompress}   %% With 'acmnumeric', to disable automatic
                            %% compression of sequential references within a
                            %% single citation;
                            %% e.g., \cite{Baker12,Baker14,Baker16}
                            %% rendered as [2,3,4] rather than [2-4].

%% Some recommended packages.
\usepackage{booktabs}   %% For formal tables:
                        %% http://ctan.org/pkg/booktabs
\usepackage{subcaption} %% For complex figures with subfigures/subcaptions
                        %% http://ctan.org/pkg/subcaption

%% Cyrus packages
% \usepackage{microtype}
% \usepackage{mdframed}
% \usepackage{colortab}
\usepackage{mathpartir}
\usepackage[normalem]{ulem}
% \usepackage{enumitem}
% \usepackage{bbm}
\usepackage{stmaryrd}
% \usepackage{mathtools}
% \usepackage{leftidx}
\usepackage{todonotes}
\usepackage{xspace}
% \usepackage{wrapfig}
\usepackage{enumitem}
\usepackage{framed}

\newcommand{\cyrus}[1]{{\color{blue} #1}}

\usepackage{listings}%
\lstloadlanguages{ML}
\lstset{tabsize=2,
basicstyle=\footnotesize\ttfamily,
% keywordstyle=\sffamily,
commentstyle=\itshape\ttfamily\color{gray},
stringstyle=\ttfamily\color{purple},
mathescape=false,escapechar=\#,
numbers=left, numberstyle=\scriptsize\color{gray}\ttfamily, language=ML, showspaces=false,showstringspaces=false,xleftmargin=15pt,
morekeywords={string, float, int, Int, Float, String, livelit, at, SpliceRef, UpdateCmd, ViewCmd, Html, return,
context, Typ, Exp, Maybe, List, Result, Dim, Unit},
classoffset=0,belowskip=\smallskipamount, aboveskip=\smallskipamount,
xleftmargin=0cm,
moredelim=**[is][\color{red}]{SSTR}{ESTR}
}
\newcommand{\li}[1]{\lstinline[basicstyle=\ttfamily\fontsize{9pt}{1em}\selectfont]{#1}}
\newcommand{\lismall}[1]{\lstinline[basicstyle=\ttfamily\fontsize{9pt}{1em}\selectfont]{#1}}

%% Joshua Dunfield macros
\def\OPTIONConf{1}%
\usepackage{joshuadunfield}

%% Can remove this eventually
% \usepackage{blindtext}

% \usepackage{enumitem}

%%%%%%%%%%%%%%%%%%%%%%%%%%%%%%%%%%%%%%%%%%%%%%%%%%%%%%%%%%%%%%%%%%%%%%%%%%%%%
%% Matt says: Cyrus, this package `adjustbox` seems directly related
%% to the `clipbox` error; To get rid of the error, I moved it last
%% (after other usepackages) and I added the line just above it, which
%% permits it to redefine `clipbox` (apparently also defined in
%% `pstricks`, and due to latex's complete lack of namespace
%% management, these would otherwise names clash).
% \let\clipbox\relax
% \usepackage[export]{adjustbox}% http://ctan.org/pkg/adjustbox
%%%%%%%%%%%%%%%%%%%%%%%%%%%%%%%%%%%%%%%%%%%%%%%%%%%%%%%%%%%%%%%%%%%%%%%%%%%%%%%%%


%%%%%%%%%%%%%%%%%%%%%%%%%%%%%%%%%%%%%%%%%%%%%%%%%%%%%%%%%%%%%%%%%%%%%%%%%%%%%%%%%
%\usepackage{draftwatermark}
%\SetWatermarkText{DRAFT}
%\SetWatermarkScale{1}
%%%%%%%%%%%%%%%%%%%%%%%%%%%%%%%%%%%%%%%%%%%%%%%%%%%%%%%%%%%%%%%%%%%%%%%%%%%%%%%%%


% A macro for the name of the system being described by ``this paper''
\newcommand{\HazelnutLive}{\textsf{Hazelnut Live}\xspace}
\newcommand{\Hazelnut}{\textsf{Hazelnut}\xspace}
% The mockup, work-in-progress system.
\newcommand{\Hazel}{\textsf{Hazel}\xspace}

% \newtheorem{theorem}{Theorem}[chapter]
% \newtheorem{lemma}[theorem]{Lemma}
% \newtheorem{corollary}[theorem]{Corollary}
% \newtheorem{definition}[theorem]{Definition}
% \newtheorem{assumption}[theorem]{Assumption}
% \newtheorem{condition}[theorem]{Condition}

\newtheoremstyle{slplain}% name
  {.15\baselineskip\@plus.1\baselineskip\@minus.1\baselineskip}% Space above
  {.15\baselineskip\@plus.1\baselineskip\@minus.1\baselineskip}% Space below
  {\slshape}% Body font
  {\parindent}%Indent amount (empty = no indent, \parindent = para indent)
  {\bfseries}%  Thm head font
  {.}%       Punctuation after thm head
  { }%      Space after thm head: " " = normal interword space;
        %       \newline = linebreak
  {}%       Thm head spec
\theoremstyle{slplain}
\newtheorem{thm}{Theorem}  % Numbered with the equation counter
\numberwithin{thm}{section}
\newtheorem{defn}[thm]{Definition}
\newtheorem{lem}[thm]{Lemma}
\newtheorem{prop}[thm]{Proposition}
% \newtheorem{cor}[section]{Corollary}
% \newtheorem{lem}[section]{Lemma}
% \newtheorem{prop}[section]{Proposition}

% \setlength{\abovedisplayskip}{0pt}
% \setlength{\belowdisplayskip}{0pt}
% \setlength{\abovedisplayshortskip}{0pt}
% \setlength{\belowdisplayshortskip}{0pt}



\makeatletter\if@ACM@journal\makeatother
%% Journal information (used by PACMPL format)
%% Supplied to authors by publisher for camera-ready submission
% \acmJournal{PACMPL}
% \acmVolume{1}
% \acmNumber{1}
% \acmArticle{1}
% \acmYear{2018}
% \acmMonth{3}
% \acmDOI{10.1145/nnnnnnn.nnnnnnn}
\startPage{1}
\else\makeatother
%% Conference information (used by SIGPLAN proceedings format)
%% Supplied to authors by publisher for camera-ready submission
% \acmConference[]{ACM SIGPLAN Conference on Programming Languages}{January 01--03, 2017}{New York, NY, USA}

% \acmYear{2018}
% \acmISBN{978-x-xxxx-xxxx-x/YY/MM}
% \acmDOI{10.1145/nnnnnnn.nnnnnnn}
% \startPage{1}
\fi


%% Copyright information
%% Supplied to authors (based on authors' rights management selection;
%% see authors.acm.org) by publisher for camera-ready submission
% \setcopyright{none}             %% For review submission
%\setcopyright{acmcopyright}
%\setcopyright{acmlicensed}
%\setcopyright{rightsretained}
%\copyrightyear{2017}           %% If different from \acmYear


\fancyfoot{} % suppresses the footer (also need \thispagestyle{empty} after \maketitle below)


%% Bibliography style
\bibliographystyle{ACM-Reference-Format}
%% Citation style
%% Note: author/year citations are required for papers published as an
%% issue of PACMPL.
% \citestyle{acmauthoryear}   %% For author/year citations

% !TEX root = main.tex

\newcommand{\mynote}[3]{\textcolor{#3}{\textsf{{#2}}}}
\newcommand{\rkc}[1]{\mynote{rkc}{#1}{blue}}
\newcommand{\cy}[1]{\mynote{cy}{#1}{purple}}
\newcommand{\mah}[1]{\mynote{cy}{#1}{green}}
\newcommand{\matt}[1]{{\color{blue}{\textit{Matt:~#1}}}}

\newcommand{\cvert}{{\,{\vert}\,}}

%% https://tex.stackexchange.com/questions/9796/how-to-add-todo-notes
\newcommand{\rkcTodo}[1]{\todo[linecolor=blue,backgroundcolor=blue!25,bordercolor=blue]{#1}}

\newcommand{\mattTodo}[1]{\todo[linecolor=green,backgroundcolor=green!2,bordercolor=green]{\tiny\textit{#1}}}
\newcommand{\mattOmit}[1]{\colorbox{yellow}{(Matt omitted stuff here)}}

% \usepackage{amssymb}% http://ctan.org/pkg/amssymb
\usepackage{pifont}% http://ctan.org/pkg/pifont
\newcommand{\cmark}{\ding{51}}%
\newcommand{\xmark}{\ding{55}}%

\def\parahead#1{\paragraph{\textbf{#1.}}}
%% \def\paraheadNoDot#1{\paragraph{{\textbf{#1}}}}
\def\subparahead#1{\paragraph{\textit{#1.}}}
%% \def\paraheadindent#1{\paragraph{}\textit{#1.}}
%% \def\paraheadindentnodot#1{\paragraph{}\textit{#1}}

% \newcommand{\ie}{{\emph{i.e.}}}
% \newcommand{\eg}{{\emph{e.g.}}}
% \newcommand{\etc}{{\emph{etc.}}}
% \newcommand{\cf}{{\emph{cf.}}}
% \newcommand{\etal}{{\emph{et al.}}}

%% \newcommand{\hazel}{\ensuremath{\textsc{Hazel}}}
%% \newcommand{\sns}{\ensuremath{\textsc{Sketch-n-Sketch}}}
%% \newcommand{\deuce}{\ensuremath{\textsc{Deuce}}}
\newcommand{\Elm}{\ensuremath{\textsf{Elm}}}
\newcommand{\sns}
  %% {\ensuremath{\textrm{Sketch-n-Sketch}}}
  {\ensuremath{\textsf{Sketch-n-Sketch}}}
\newcommand{\deuce}
  %% {\ensuremath{\textrm{Deuce}}}
  {\ensuremath{\textsf{Deuce}}}

\newcommand{\sectionDescription}[1]{\section{#1}}
\newcommand{\subsectionDescription}[1]{\subsection{#1}}
\newcommand{\subsubsectionDescription}[1]{\subsubsection{#1}}
%% \newcommand{\subsectionDescription}[1]{\subsection*{#1}}
\newcommand{\suppMaterials}{the Supplementary Materials}

\newcommand{\defeq}{\overset{\textrm{def}}{=}}

\newcommand{\eap}{action suggestion panel\xspace}
\newcommand{\Eap}{Action suggestion panel\xspace}

\newcommand{\myfootnote}[1]{\footnote{ #1}}

\def\sectionautorefname{Section}
\def\subsectionautorefname{Section}
\def\subsubsectionautorefname{Section}

\newcommand{\code}[1]{\lstinline{#1}}

% Make italic?
%\newcommand{\Property}[1]{\emph{#1}}
\newcommand{\Property}[1]{\textrm{#1}}

% Calling out Cyrus's favorite verb, 'to be' ;)
\newcommand{\IS}{\colorbox{red}{is}\xspace}

\newcommand{\codeSize}
  %% {\footnotesize}
  {\small}

%\newcommand{\JoinTypes}[2]{\textsf{join}~~#1~~#2}
\newcommand{\JoinTypes}[2]{\textsf{join}(#1,#2)}

%%%%%%%%%%%%%%%%%%%%%%%%%%%%%%%%%%%%%%%%%%%%%%%%%%%%%%%%%%%%%%%%%%%%%%%%%%%%%%%%
%% Spacing

\newcommand{\sep}{\hspace{0.06in}}
\newcommand{\sepPremise}{\hspace{0.20in}}
\newcommand{\hsepRule}{\hspace{0.20in}}
\newcommand{\vsepRuleHeight}{0.08in}
\newcommand{\vsepRule}{\vspace{\vsepRuleHeight}}
\newcommand{\miniSepOne}{\hspace{0.01in}}
\newcommand{\miniSepTwo}{\hspace{0.02in}}
\newcommand{\miniSepThree}{\hspace{0.03in}}
\newcommand{\miniSepFour}{\hspace{0.04in}}
\newcommand{\miniSepFive}{\hspace{0.05in}}

%%%%%%%%%%%%%%%%%%%%%%%%%%%%%%%%%%%%%%%%%%%%%%%%%%%%%%%%%%%%%%%%%%%%%%%%%%%%%%%%

% \lstset{
% %mathescape=true,basicstyle=\fontsize{8}{9}\ttfamily,
% literate={=>}{$\Rightarrow$}2
%          {<=}{$\leq$}2
%          {->}{${\rightarrow}$}1
%          {\\\\=}{\color{red}{$\lambda$}}2
%          {\\\\}{$\lambda$}2
%          {**}{$\times$}2
%          {*.}{${\color{blue}{\texttt{*.}}}$}2
%          {+.}{${\color{blue}{\texttt{+.}}}$}2
%          {<}{${\color{green}{\lhd}}$}1
%          {>?}{${\color{green}{\rhd}}$?}2
%          {<<}{${\color{green}{\blacktriangleleft}}$}1
%          {>>?}{${\color{green}{\blacktriangleright}}$?}2
%          {\{}{${\color{blue}{\{}}$}1
%          {\}}{${\color{blue}{\}}}$}1
%          {[}{${\color{purple}{[}}$}1
%          {]}{${\color{purple}{]}}$}1
%          {(}{${\color{darkgray}{\texttt{(}}}$}1
%          {)}{${\color{darkgray}{\texttt{)}}}$}1
%          {]]}{${\color{gray}{\big(}}$}1
%          {]]}{${\color{gray}{\big)}}$}1
% }

% !TEX root = livelits-paper.tex

% \newcommand{\Label}[1]{\vspace{-20px}\label{#1}%
%   {\small\textcolor{cyan}{(\texttt{#1})}}\vspace{20px}%
% }

\newcommand{\cmttclo}[2]{\mathsf{clo}(#1, #2)}

% \newcommand{\CaptionLabel}[2]{
%   \caption{#1 {\small\textcolor{cyan}{(#2)}}}
%   \label{#2}}
\newcommand{\CaptionLabel}[2]{
  \caption{#1}
  \label{#2}}

% Violet hotdogs; highlight color helps distinguish them
\newcommand{\llparenthesiscolor}{\textcolor{violet}{\llparenthesis}}
\newcommand{\rrparenthesiscolor}{\textcolor{violet}{\rrparenthesis}}
% \newcommand{\llparenthesiscolor}{\textcolor{red}{\lfloor}}
% \newcommand{\rrparenthesiscolor}{\textcolor{red}{\rfloor}}

\newcommand{\fmap} [3] {\{ #1 ~ \rotatebox[origin=c]{180}{$\Lsh$} ~ #3 \}_{#1 \in #2}}

%% TODO if feeling really obsessive, use the following in place of x,u,c,b
\newcommand{\varVar}{x}
\newcommand{\varHole}{u}
\newcommand{\econst}{c}
\newcommand{\tbase}{b}

% HTyp and HExp
\newcommand{\isComplete}[1]{#1~\mathsf{complete}}

% HTyp
\newcommand{\htyp}[0]{\tau} % Meta variable for types
\newcommand{\tarr}[2]{#1 \rightarrow #2}
%\newcommand{\tsum}[2]{#1 + #2}
\newcommand{\tprod}[2]{#1 \times #2}
\newcommand{\tunit}{\mathsf{1}}
\newcommand{\tnum}{\mathsf{num}}
\newcommand{\tb}{\texttt{b}}
\newcommand{\tehole}{\llparenthesiscolor\rrparenthesiscolor}
\newcommand{\tsum}[2]{{#1} + {#2}}
\newcommand{\trec}[2]{\mu(#1.#2)}

\newcommand{\tconsistent}[2]{#1 \sim #2}
\newcommand{\tinconsistent}[2]{#1 \nsim #2}
\newcommand{\tconsistentc}[2]{#1 \sim_c #2}

% Expression forms
\newcommand{\hexp}{e} % Hexp without palettes
\newcommand{\pexp}{p} % Meta variable for palette expressions
\newcommand{\encExp}{\mathtt{Exp}} % Whatever htyp encoded things should have

% HExp
\newcommand{\hlam}[2]{\lambda #1.#2}
\newcommand{\halam}[3]{\lambda #1{:}#2.#3}
\newcommand{\hap}[2]{#1~#2}
\newcommand{\hapP}[2]{(#1)~(#2)} % Extra paren around function term
\newcommand{\hpair}[2]{(#1, #2)}
\newcommand{\hprj}[2]{\mathsf{prj}_{#1}(#2)}
\newcommand{\hprl}[1]{\mathsf{prl}(#1)}
\newcommand{\hprr}[1]{\mathsf{prr}(#1)}
\newcommand{\htriv}{()}
\newcommand{\lblL}{\mathsf{L}}
\newcommand{\lblR}{\mathsf{R}}
\newcommand{\hnum}[1]{\underline{#1}}
\newcommand{\helet}[3]{\mathsf{let}~#1=#2~\mathsf{in}~#3}
%\newcommand{\hcase}[5]{\mathsf{case}\,#1\,\mathsf{of}\,#2\Rightarrow#3~\vert~#4\Rightarrow#5}
\newcommand{\hadd}[2]{#1 + #2}
\newcommand{\hehole}[1]{\llparenthesiscolor\rrparenthesiscolor^{#1}}
% \newcommand{\hhole}[1]{\setlength{\fboxsep}{0pt}\fcolorbox{red}{white}{\vphantom{)}$#1$}}
\newcommand{\hhole}[2]{\llparenthesiscolor#1\rrparenthesiscolor^{#2}}
% \newcommand{\hhole}[1]{
  % \setlength{\fboxsep}{0pt}
  % \colorbox{violet!10!white!100}{\ensuremath{\llparenthesiscolor#1\rrparenthesiscolor}}}
\newcommand{\hindet}[1]{\lceil#1\rceil}
%\newcommand{\hinj}[2]{\texttt{inj}_{#1}({#2})}
\newcommand{\hinL}[1]{\mathsf{inl}(#1)}
\newcommand{\hinR}[1]{\mathsf{inr}(#1)}
\newcommand{\hcase}[5]{\texttt{case}({#1},{#2}.{#3},{#4}.{#5})}
\newcommand{\hroll}[1]{\mathsf{roll}(#1)}
\newcommand{\hunroll}[1]{\mathsf{unroll}(#1)}
\newcommand{\livelitname}[1]{\$#1}
\newcommand{\haplivelit}[4]{\livelitname{#2}\langle #3; #4 \rangle^{#1}}
\newcommand{\hsplice}[2]{#1 : #2}

% Palettes
\newcommand{\pexpPalLet}[3]{\keyword{let palette}\,#1\,\keyword{=}\,#2\,\keyword{in}\,#3}
\newcommand{\pexpPalAp}[4]{#1 \left(#2;\,#4 : #3 \right)}
\newcommand{\pexpPalApF}[3]{#1 \left(#2;\,#3\right)}

% Palette definitions
\newcommand{\pDef}{\pi} % Meta variable for palette definitions
\newcommand{\pDefRecord}[3]{\{ \mathsf{expand}: #1, \, \mathsf{modelTyp}: #2, \, \mathsf{expandTyp}: #3\}}
\newcommand{\pDefRecordF}[4]{\{ \mathsf{expand}: #1, \, \mathsf{modelTyp}: #2, \, \mathsf{spliceTyp}: #3, \, \mathsf{expandTyp}: #4\}}
\newcommand{\pPhiWF}[1]{#1 \, \, \mathsf{palctx}}

\newcommand{\hGamma}{\Gamma}
\newcommand{\pPhi}{\Phi}
\newcommand{\EmptyhGamma}{\emptyset}
\newcommand{\EmptyDelta}{\emptyset}
\newcommand{\domof}[1]{\text{dom}(#1)}
\newcommand{\hsyn}[3]{#1 \vdash #2 \Rightarrow #3}
\newcommand{\hana}[3]{#1 \vdash #2 \Leftarrow #3}

% ZTyp and ZExp
\newcommand{\zlsel}[1]{{\bowtie}{#1}}
\newcommand{\zrsel}[1]{{#1}{\bowtie}}

%\newcommand{\zwsel}[1]{\adjustbox{cframe=blue}{\ensuremath{{\textcolor{blue}{\triangleright}}{#1}{\textcolor{blue}{\triangleleft}}}}}
\newcommand{\zwsel}[1]{
  \setlength{\fboxsep}{0pt}
  \colorbox{green!10!white!100}{
    \ensuremath{{{\textcolor{Green}{{\hspace{-2px}\triangleright}}}}{#1}{\textcolor{Green}{\triangleleft{\vphantom{\tehole}}}}}}
}
%\newcommand{\zwsel}[1]{{\triangleright}{#1}{\triangleleft}}

\newcommand{\removeSel}[1]{#1^{\diamond}}

% ZTyp
\newcommand{\ztau}{\hat{\tau}}

% ZExp
\newcommand{\zexp}{\hat{e}}

% Direction
\newcommand{\dParent}{\mathtt{parent}}
\newcommand{\dChild}{\mathtt{firstChild}}
\newcommand{\dNext}{\mathtt{nextSib}}
\newcommand{\dPrev}{\mathtt{prevSib}}

% Action
\newcommand{\aMove}[1]{\mathtt{move}~#1}
	\newcommand{\zrightmost}[1]{\mathsf{rightmost}(#1)}
	\newcommand{\zleftmost}[1]{\mathsf{leftmost}(#1)}
\newcommand{\aSelect}[1]{\mathtt{sel}~#1}
\newcommand{\aDel}{\mathtt{del}}
\newcommand{\aReplace}[1]{\mathtt{replace}~#1}
\newcommand{\aConstruct}[1]{\mathtt{construct}~#1}
\newcommand{\aConstructx}[1]{#1}
\newcommand{\aFinish}{\mathtt{finish}}

\newcommand{\performAna}[5]{#1 \vdash #2 \xlongrightarrow{#4} #5 \Leftarrow #3}
\newcommand{\performAnaI}[5]{#1 \vdash #2 \xlongrightarrow{#4}\hspace{-3px}{}^{*}~ #5 \Leftarrow #3}
\newcommand{\performSyn}[6]{#1 \vdash #2 \Rightarrow #3 \xlongrightarrow{#4} #5 \Rightarrow #6}
\newcommand{\performSynI}[6]{#1 \vdash #2 \Rightarrow #3 \xlongrightarrow{#4}\hspace{-3px}{}^{*}~ #5 \Rightarrow #6}
\newcommand{\performTyp}[3]{#1 \xlongrightarrow{#2} #3}
\newcommand{\performTypI}[3]{#1 \xlongrightarrow{#2}\hspace{-3px}{}^{*}~#3}

\newcommand{\performMove}[3]{#1 \xlongrightarrow{#2} #3}
\newcommand{\performDel}[2]{#1 \xlongrightarrow{\aDel} #2}

% Form
\newcommand{\farr}{\mathtt{arrow}}
\newcommand{\fnum}{\mathtt{num}}
\newcommand{\fsum}{\mathtt{sum}}

\newcommand{\fasc}{\mathtt{asc}}
\newcommand{\fvar}[1]{\mathtt{var}~#1}
\newcommand{\flam}[1]{\mathtt{lam}~#1}
\newcommand{\fap}{\mathtt{ap}}
\newcommand{\farg}{\mathtt{arg}}
\newcommand{\fnumlit}[1]{\mathtt{lit}~#1}
\newcommand{\fplus}{\mathtt{plus}}
\newcommand{\fhole}{\mathtt{hole}}
\newcommand{\fnehole}{\mathtt{nehole}}

\newcommand{\finj}[1]{\mathtt{inj}~#1}
\newcommand{\fcase}[2]{\mathtt{case}~#1~#2}

% Talk about formal rules in example
\newcommand{\refrule}[1]{\textrm{Rule~(#1)}}

\newcommand{\herase}[1]{\left|#1\right|_\textsf{erase}}

\newcommand{\arrmatch}[2]{#1 \blacktriangleright_{\rightarrow} #2}
%% TODO maybe write underbracket
%% \newcommand{\groundmatch}[2]{\underline{#1} = #2}
\newcommand{\groundmatch}[2]{#1 \blacktriangleright_{\mathsf{ground}} #2}
\newcommand{\prodmatch}[2]{#1 \blacktriangleright_{\times} #2}
\newcommand{\summatch}[2]{#1 \blacktriangleright_{+} #2}


\newcommand{\TABperformAna}[5]{#1 \vdash & #2                & \xlongrightarrow{#4} & #5 & \Leftarrow #3}
\newcommand{\TABperformSyn}[6]{#1 \vdash & #2 \Rightarrow #3 & \xlongrightarrow{#4} & #5 \Rightarrow #6}
\newcommand{\TABperformTyp}[3]{& #1 & \xlongrightarrow{#2} & #3}

\newcommand{\TABperformMove}[3]{#1 & \xlongrightarrow{#2} & #3}
\newcommand{\TABperformDel}[2]{#1 \xlongrightarrow{\aDel} #2}

\newcommand{\sumhasmatched}[2]{#1 \mathrel{\textcolor{black}{\blacktriangleright_{+}}} #2}

%%%% DYNAMICS %%%%
%% TODO remove these macros
%% marks for eval
\newcommand{\unevaled}{\times}
\newcommand{\evaled}{\checkmark}
\newcommand{\markname}{m}

\newcommand{\mvar}[0]{u}
\newcommand{\subst}[0]{\sigma}
\newcommand{\substitute}[3]{[#1/#2]#3}
\newcommand{\substitutesub}[4]{[#1/#2]_{#3}\,#4}
\newcommand{\fvof}[1]{\mathsf{FV}(#1)}
\newcommand{\dexp}[0]{d}
\newcommand{\dconst}[0]{c}
\newcommand{\dval}[0]{\ddot{v}}
%% TODO remove this macro
\newcommand{\dcast}[2]{\langle #1 \rangle ~ #2}
%% TODO make the following two look better
\newcommand{\dcasttwo}[3]{#1 \langle{#2}\Rightarrow{#3}\rangle}
\newcommand{\dcastthree}[4]
  {#1 \langle{#2}\Rightarrow{#3}\Rightarrow{#4}\rangle} %% sugared version
  %% {\dcasttwo{\dcasttwo{#1}{#2}{#3}}{#3}{#4}} %% unsugared version
\newcommand{\dcastfail}[3]{#1 \langle{#2}\Rightarrow{\tehole}\not\Rightarrow{#3}\rangle}
%% \newcommand{\dlam}[3]{\lambda #1:#2.#3}
\newcommand{\dlam}[3]{\halam{#1}{#2}{#3}}
\newcommand{\dap}[2]{#1(#2)}
\newcommand{\dapP}[2]{(#1)(#2)} % Extra paren around function term
\newcommand{\dnum}[1]{\underline{#1}}
%\newcommand{\dcase}[5]{\mathsf{case}\,#1\,\mathsf{of}\,#2\Rightarrow#3~\vert~#4\Rightarrow#5}
\newcommand{\dadd}[2]{#1 + #2}
%% TODO third arg should be empty
\newcommand{\dehole}[2]{{\llparenthesiscolor\rrparenthesiscolor}{^{#1}_{#2}}}
%% TODO fourth arg should be empty
\newcommand{\dhole}[4]{\leftidx{^{#4}}{\llparenthesiscolor#1\rrparenthesiscolor}{^{#2}_{#3}}}
\newcommand{\dindet}[1]{\lceil#1\rceil}
%\newcommand{\dinj}[2]{\texttt{inj}_{#1}({#2})}
\newcommand{\dinL}[2]{\mathsf{inl}_{#1}(#2)}
\newcommand{\dinR}[2]{\mathsf{inr}_{#1}(#2)}
\newcommand{\dcase}[5]{\texttt{case}({#1},{#2}.{#3},{#4}.{#5})}
\newcommand{\dpair}[2]{(#1,#2)}
\newcommand{\dprj}[2]{\mathsf{prj}_{#1}(#2)}

\newcommand{\isType}[2]{#1 \vdash #2~\mathtt{type}}
\newcommand{\elabs}[5]{#1 \vdash #2 \leadsto #3 : #4 \dashv #5}
\newcommand{\expands}[5]{#1; #2 \vdash #3 \leadsto #4 : #5}
\newcommand{\ccexpands}[6]{#1; #2 \vdash_\text{cc} #3 \leadsto #4 : #5 \dashv #6}
\newcommand{\Omegaitem}[2]{#1 \hookrightarrow #2}
\newcommand{\expandAna}[6]{#1 \vdash #2 \Leftarrow #3 \leadsto #4 : #5 \dashv #6}
\newcommand{\expandSyn}[5]{#1 \vdash #2 \Rightarrow #3 \leadsto #4 \dashv #5}
\newcommand{\pexpandAna}[5]{#1 \vdash_{#2} #3 \leadsto #4 \Leftarrow #5}
\newcommand{\pexpandSyn}[5]{#1 \vdash_{#2} #3 \leadsto #4 \Rightarrow #5}
\newcommand{\decodeExp}[2]{#1 \uparrow #2}
\newcommand{\encodeExp}[2]{#1 \downarrow #2}
\newcommand{\hasType}[3]{#1 \vdash #2 : #3}
\newcommand{\hasTypeD}[4]{#1; #2 \vdash #3 : #4}
\newcommand{\isValue}[1]{#1~\mathsf{val}}
\newcommand{\isGround}[1]{#1~\mathsf{ground}}
\newcommand{\isBoxedValue}[1]{#1~\mathsf{boxedval}}
\newcommand{\isIndet}[1]{#1~\mathsf{indet}}
\newcommand{\isFinal}[1]{#1~\mathsf{final}}
\newcommand{\isErr}[2]{#1 \vdash #2~\mathsf{err}}
%% \newcommand{\stepsTo}[2]{#1 \mapsto_{\Delta} #2}
%% TODO first arg should be empty
%% \newcommand{\stepsToD}[3]{#1 \vdash #2 \mapsto #3}
\newcommand{\stepsToD}[3]{#2 \mapsto #3}
\newcommand{\multiStepsTo}[2]{#1 \mapsto^* #2}
\newcommand{\evalsTo}[2]{#1 \Downarrow #2}

%% TODO if feeling obsessive, replace direct uses of \Delta
\newcommand{\hDelta}{\Delta}
\newcommand{\Dunion}[2]{#1 \cup #2}
\newcommand{\idof}[1]{\mathsf{id}(#1)}
\newcommand{\Dbinding}[3]{#1 :: #3[#2]}
\newcommand{\instantiate}[3]{\llbracket#1 / #2\rrbracket #3}
\newcommand{\instantiateB}[2]{\llbracket #1 / #2\rrbracket}

% Contextual dynamics
\newcommand{\evalctx}{\mathcal{E}}
\newcommand{\evalhole}{\circ}
\newcommand{\isevalctx}[1]{#1~\mathsf{evalCtx}}
%% TODO first arg should be empty
%% \newcommand{\reducesE}[3]{#1 \vdash #2 \longrightarrow #3}
\newcommand{\reducesE}[3]{#2 \longrightarrow #3}
\newcommand{\selectEvalCtxR}[2]{#1\{#2\}}
\newcommand{\selectEvalCtx}[3]{#1=\selectEvalCtxR{#2}{#3}}
\newcommand{\maybePremise}[1]{{\textcolor{red}[}#1{\textcolor{red}]}}

\newcommand{\inhole}[2]{\mathsf{inhole}(#1; #2)}

\newcommand{\DoSubst}[3]{[#1/#2]{#3}}

\newcommand{\splat}[1]{\{#1\}_{i < n}}
\newcommand{\taut}[1]{\tau_\text{#1}}
\newcommand{\etxt}[1]{e_\text{#1}}
\newcommand{\dtxt}[1]{d_\text{#1}}
\newcommand{\splices}{\splat{\psi_i}}

\newcommand{\livelitCtxEntry}[4]{\mathsf{livelit}~\livelitname{#1}~\mathsf{at}~#2~\{#3; #4\}}
\newcommand{\expType}{\mathsf{Exp}}
\newcommand{\livelitsIn}[1]{\mathsf{livelits}(#1)}
\newcommand{\protoEnvsOf}[3]{\mathsf{protoenvs}_{#1}(#2; #3)}
\newcommand{\envsOf}[3]{\mathsf{envs}_{#1}(#2; #3)}
\newcommand{\fillof}[2]{\mathsf{fill}_{#1}(#2)}
\newcommand{\resumeof}[1]{\mathsf{resume}(#1)}


\setlength{\abovecaptionskip}{4pt plus 6pt minus 6pt} % Chosen fairly arbitrarily
\setlength{\belowcaptionskip}{-4pt plus 6pt minus 6pt} % Chosen fairly arbitrarily


\begin{document}

%% Title information
\title
  [Hazel Assistant Calculus WIP]
  {Hazel Assistant Calculus WIP}

                                        %% when present, will be used in
                                        %% header instead of Full Title.
% \titlenote{with title note}             %% \titlenote is optional;
                                        %% can be repeated if necessary;
                                        %% contents suppressed with 'anonymous'
% \subtitle{Subtitle}                     %% \subtitle is optional
% \subtitlenote{with subtitle note}       %% \subtitlenote is optional;
                                        %% can be repeated if necessary;
                                        %% contents suppressed with 'anonymous'


%% Author information
%% Contents and number of authors suppressed with 'anonymous'.
%% Each author should be introduced by \author, followed by
%% \authornote (optional), \orcid (optional), \affiliation, and
%% \email.
%% An author may have multiple affiliations and/or emails; repeat the
%% appropriate command.
%% Many elements are not rendered, but should be provided for metadata
%% extraction tools.

\author{Andrew Blinn}
% \authornote{with author1 note}          %% \authornote is optional;
                                        %% can be repeated if necessary
% \orcid{nnnn-nnnn-nnnn-nnnn}             %% \orcid is optional
\affiliation{
  % \position{Position1}
  % \department{Department1}              %% \department is recommended
  \institution{University of Michigan, USA}            %% \institution is required
  % \streetaddress{Street1 Address1}
  % \city{City1}
  % \state{State1}
  % \postcode{Post-Code1}
  % \country{Country1}
}
\email{blinnand@umich.edu}          %% \email is recommended

%% Author with single affiliation.
\author{Cyrus Omar}
% \authornote{with author1 note}          %% \authornote is optional;
                                        %% can be repeated if necessary
% \orcid{nnnn-nnnn-nnnn-nnnn}             %% \orcid is optional
\affiliation{
  % \position{Position1}
  % \department{Department1}              %% \department is recommended
  \institution{University of Michigan, USA}            %% \institution is required
  % \streetaddress{Street1 Address1}
  % \city{City1}
  % \state{State1}
  % \postcode{Post-Code1}
  % \country{Country1}
}
\email{comar@umich.edu}          %% \email is recommended

%% Paper note
%% The \thanks command may be used to create a "paper note" ---
%% similar to a title note or an author note, but not explicitly
%% associated with a particular element.  It will appear immediately
%% above the permission/copyright statement.
% \thanks{with paper note}                %% \thanks is optional
                                        %% can be repeated if necesary
                                        %% contents suppressed with 'anonymous'


%% Abstract
%% Note: \begin{abstract}...\end{abstract} environment must come
%% before \maketitle command
% !TEX root = livelits-paper.tex

%% no citations in Abstract
%%
%% \cite{popl-paper}

\begin{abstract}
    The hazelnut assistant calculus provides an extensible framework for type- and value-directed completion and refactoring support in a structured editing context.
\end{abstract}



%% 2012 ACM Computing Classification System (CSS) concepts
%% Generate at 'http://dl.acm.org/ccs/ccs.cfm'.
% \begin{CCSXML}
% <ccs2012>
% <concept>
% <concept_id>10011007.10011006.10011008</concept_id>
% <concept_desc>Software and its engineering~General programming languages</concept_desc>
% <concept_significance>500</concept_significance>
% </concept>
% <concept>
% <concept_id>10003456.10003457.10003521.10003525</concept_id>
% <concept_desc>Social and professional topics~History of programming languages</concept_desc>
% <concept_significance>300</concept_significance>
% </concept>
% </ccs2012>
% \end{CCSXML}

\ccsdesc[500]{Software and its engineering~General programming languages}
% \ccsdesc[300]{Social and professional topics~History of programming languages}
%% End of generated code


%% Keywords
%% comma separated list
% \keywords{keyword1, keyword2, keyword3}  %% \keywords is optional
\keywords{live programming, code completion, refactoring, GUIs}


%% \maketitle
%% Note: \maketitle command must come after title commands, author
%% commands, abstract environment, Computing Classification System
%% environment and commands, and keywords command.
\maketitle
% \thispagestyle{empty} % suppresses the footer


wazzzzzaaaaaaaaaaaaaaaaaaaaaa

\section{Assistant Calculus}\label{sec:assistant-calculus}

blah blah blah types

TODOs:
\begin{itemize}
    \item \sout{get cursor icons from hazelnut paper}
    \item \sout{get right arrow for bidi}
    \item \sout{basic zipper cases}
    \item remaining zipper cases? do i need to actually include mirror cases?
    \item \sout{var + varapp}
    \item \sout{NOTE: we basically need a construct expression action for varapp}
    \item \sout{proj}
    \item \sout{base case for hole}
    \item \sout{base cases for non-empty holes, incld:}
    \item \sout{delete + act for general hexps}
    \item \sout{simple wrap for exprs incld. non-empty-holes}
    \item complex (n-ary) wraps
    \item iterated wraps? with cutoffs? (:jean-shorts-emoji)
    \item \sout{are there non-empty-hole suggests distinct from arbitrary expr suggests?} don't think so
    \item \sout{for all: change type to type consistency}
    \item \sout{for all: add numerical subscripts to types where missing}
    \item \sout{fig 4: make it consistency not equal}
    \item \sout{fig 5: change proj judgement to analysis}
    \item \sout{fig 4: change varapp proj to have x in gamma, not gamma comma x}
    \item \sout{consider matched product, matched arrow to suggest for unknown types}
    \item above: not sure i want matched arrow type in fig 5? feels weirder to suggest an unknown typed var for fn than value...
    \item rankings:
    \item priviledge more specific types
    \item read contextual modal types
    \item \sout{replace constructs with construct-expressions}
    \item AppProj needs a better treatment for selection (should be first non-empty hole)... chain a separate action?
    \item implementation: implement ENTER vs TAB
    \item add keyboard shortcuts for swap etc
    \item proj1 and proj2: make tau in product another underscore
    \item intro rule for type=hole
    \item fig 4 var and appproj case: type consistency not premise: put in 
    \item read: polarity: noam zalburger. bob harper blog post
    \item what does it mean to synthesize 'action a': sensibility theorem:
    \item well-typed in tsynthetid case, checked against tau in analytic case 
\end{itemize}

\newcommand{\singleton}[1]{\{ #1 \}}
\newcommand{\singleaction}[1]{\singleton{\mathtt{#1}}}
\newcommand{\varapparrow}[0]{\rightsquigarrow}
\newcommand{\varappjudge}[4]{#1 \vdash #2 \varapparrow #3 \Leftarrow #4}
\newcommand{\projjudge}[4]{#1 \vdash #2 \twoheadrightarrow #3 \Leftarrow #4}

\begin{figure}
    \begin{mathpar}

        \fbox{$\suggest
        {\hsyn{\hGamma}{\zexp}{\tau}}
        {\bigalpha}$}~~\text{$\hexp$ synthesizes $\tau$, suggesting actions $\bigalpha$} \\

        \fbox{$\suggest
        {\hana{\hGamma}{\zexp}{\tau}}
        {\bigalpha}$}~~\text{$\hexp$ analyzes against $\tau$, suggesting actions $\bigalpha$}  \\
        
        \inferrule[]{
            \hana{\hGamma}{\hexp_1}{\tau_2} \\
            \suggest
                {\hana{\hGamma}{\zexp_2}{\tau_1}}
                {\bigalpha}
        }{
            \suggest
                {\hsyn{\hGamma}{(\zexp_1, \hexp_2)}{\tprod{\tau_1}{\tau_2}}}
                {\bigalpha}
        }

        \inferrule[]{
            \suggest
                {\hsyn{\hGamma}{\zexp}{\tau_1}}
                {\bigalpha} \\
            \arrmatch{\tau_1}{\tarr{\tau_2}{\tau}} \\
            \hana{\hGamma}{\hexp_2}{\tau_2}
        }{
            \suggest
                {\hsyn{\hGamma}{\dap{\zexp_1}{\hexp_2}}{\tau}}
                {\bigalpha}
        }

        \inferrule[]{
            \arrmatch{\tau}{\tarr{\tau_1}{\tau_2}} \\
            \suggest
                {\hana{\hGamma}{\zexp}{\tau}}
                {\bigalpha}
        }{
            \suggest
                {\hana{\hGamma}{\hlam{x}{\zexp}}{\tau}}
                {\bigalpha}
        }


    \end{mathpar}
    \caption{Suggestion Zipper Cases}
    \label{fig:suggests_zipper}
\end{figure}

\begin{figure}
    \begin{mathpar}
        \inferrule[Suggest Hole Analytic]{
            \suggest{\mathsf{Intros}(\tau)}{\bigalpha_{intros}} \\
            \suggest{\mathsf{Elims}(\Gamma, \tau)}{\bigalpha_{elims}}
        }{
            \suggest
                {\hana{\hGamma}{\zwsel{\hehole{}}}{\tau}}
                {\bigalpha_{intros} \cup \bigalpha_{elims}}
        }

        \inferrule[Suggest Expr Analytic]{
            \suggest{\mathsf{Wraps}(\hexp, \tau)}{\bigalpha_{wraps}} \\
            \suggest{\mathsf{Replaces}(\Gamma, \tau)}{\bigalpha_{replaces}}
        }{
            \suggest
                {\hana{\hGamma}{\zwsel{\hexp}}{\tau}}
                {\bigalpha_{wraps} \cup \bigalpha_{replaces}}
        }

        \inferrule[Replacement]{
            \suggest
                {\hana{\hGamma}{\zwsel{\hehole{}}}{\tau}}
                {\bigalpha}
        }
        {
            \suggest
                {\mathsf{Replaces}(\Gamma, \tau)}
                {\{ \mathtt{del} \;;\; \alpha  \;|\; \alpha \in \bigalpha\}}
        }

        \inferrule[Wrapping (simple)]{
            \hsyn{\hGamma}{\hexp}{\tau_e} \\
            \tconsistent{\tau_e}{\tau'}
        }
        {
            \suggest
                {\mathsf{Wraps}(\hexp, \tau)}
                {\{ \mathtt{construct\;} \dap{f}{\zwsel{\hexp}} \;|\; f : \tarr{\tau'}{\tau} \in \Gamma\}}
        }        
    \end{mathpar}
    \caption{Suggestion base cases}
    \label{fig:suggests}
\end{figure}

\begin{figure}
    \begin{mathpar}
    
        \inferrule[IntrosTriv]
            {\tconsistent{\tau}{\tunit}}
            {\suggest
                {\mathsf{Intros}(\tau)}
                {\singleaction{construct\; \zwsel{()}}}}

        \inferrule[IntrosProd]
            {\tconsistent{\tau}{\tprod{\tau_1}{\tau_2}}}
            {\suggest
                {\mathsf{Intros}(\tau)}
                {\singleaction{construct\; \hpair{\zwsel{\hehole{}}}{\hehole{}}}}}
        
        \inferrule[IntrosArrow]
            {\tconsistent{\tau}{\tarr{\tau_1}{\tau_2}}}
            {\suggest
                {\mathsf{Intros}(\tau)}
                {\singleaction{construct\; \hlam{x}{\zwsel{\hehole{}}}}}}

        \inferrule[IntrosSum]
            {\tconsistent{\tau}{\tsum{\tau_1}{\tau_2}}}
            {\suggest
                {\mathsf{Intros}(\tau)}
                {\{ \mathtt{construct\; \hinL{\zwsel{\hehole{}}}},
                 \mathtt{construct\; \hinR{\zwsel{\hehole{}}}}} \}}

    \end{mathpar}
    \caption{Introduction suggestions}
    \label{fig:suggest_intros}
\end{figure}

\begin{figure}
    \begin{mathpar}

        \inferrule[Suggest Elims]{
            \suggest{\mathsf{ElimCase}}{\bigalpha_{case}} \\
            \suggest{\mathsf{Var}(\Gamma, \tau)}{\bigalpha_{var}} \\
            \suggest{\mathsf{AppProj}(\Gamma, \tau)}{\bigalpha_{approj}} \\
        }{
         \suggest
            {\mathsf{Elims}(\Gamma, \tau)}
            {\bigalpha_{case} \cup \bigalpha_{var} \cup \bigalpha_{approj}}
        }

        \inferrule[ElimCase]
            { }
            {\suggest
                {\mathsf{ElimCase}}
                {\singleaction{construct\; \hcase{\zwsel{\hehole{}}}{x}{\hehole{}}{y}{\hehole{}}}}}

        \inferrule[Var]
            {\tconsistent{\tau}{\tau'}}
            {\suggest
                {\mathsf{Var}(\Gamma, \tau)}
                {\{\mathtt{construct \; \zwsel{x}} \;|\; x : \tau' \in \Gamma\}}}

        \inferrule[AppProj]
            {\tconsistent{\tau}{\tau'}}
            {\suggest
                {\mathsf{AppProj}(\Gamma, \tau)}
                {\{\mathtt{construct \; \zwsel{e}}
                 \;|\;
                 x : \tau' \in \Gamma \; \wedge \; \varappjudge{\Gamma}{x}{e}{\tau}\}}}

    \end{mathpar}
    \caption{Elimination suggestions}
    \label{fig:suggest_elims}
\end{figure}

\begin{figure}
    \judgbox
        {\varappjudge{\Gamma}{\hexp}{\hexp'}{\tau}}
        { \\ $\hexp$, applied to 0 or more holes, \\ and projected 0 or more times, \\
         yields an expr analyzing to $\tau$}
    \begin{mathpar}
        \inferrule[AppProjBase]
            {\hana{\Gamma}{\hexp}{\tau}}
            {\varappjudge{\Gamma}{e}{e}{\tau}}

        \inferrule[App]
            {
                \hsyn{\Gamma}{\hexp}{\tau'} \\
                \arrmatch{\tau'}{\tarr{\_}{\_}} \\
                \varappjudge{\Gamma}{\dap{\hexp}{\hehole{}}}{\hexp'}{\tau}
            }
            {\varappjudge{\Gamma}{\hexp}{\hexp'}{\tau}}

        \inferrule[Proj1]
                {
                    \hsyn{\Gamma}{\hexp}{\tau'} \\
                    \prodmatch{\tau'}{\tprod{\tau}{\_}} \\
                    \varappjudge{\Gamma}{\pi_1 \hexp}{\hexp'}{\tau}
                }
                {\varappjudge{\Gamma}{e}{e'}{\tau}}
        
        \inferrule[Proj2]
                {
                    \hsyn{\Gamma}{\hexp}{\tau'} \\
                    \prodmatch{\tau'}{\tprod{\_}{\tau}} \\
                    \varappjudge{\Gamma}{\pi_2 \hexp}{\hexp'}{\tau}
                }
                {\varappjudge{\Gamma}{e}{e'}{\tau}}   

    \end{mathpar}
    \caption{Supporting elimination judgments}
    \label{fig:support_elims}
\end{figure}

%\section{Livelit Definitions}\label{sec:livelit-definitions}

\begin{figure}
\begin{lstlisting}[xleftmargin=0.2cm]
type Color = (.r Int, .g Int, .b Int, .a Int)
livelit $color at Color {
  type Model = (.r SpliceRef, .g SpliceRef, 
                .b SpliceRef, .a SpliceRef)

  context { }

  let init : UpdateCmd(Model) = do 
    r <- new_splice(`Int`, Some(`0`))
    g <- new_splice(`Int`, Some(`0`))
    b <- new_splice(`Int`, Some(`0`))
    a <- new_splice(`Int`, Some(`100`))
    return (r, g, b, a)

  type Action = 
  | ClickOn(Color)

  let view : Model -> ViewCmd(Html(Action)) = 
    fun model -> do 
      (* determine a color to display *)
      r_res <- eval_splice(model.r)
      g_res <- eval_splice(model.g)
      b_res <- eval_splice(model.b)
      a_res <- eval_splice(model.a)
      let cur_color : Color = 
        case (r_res, g_res, b_res, a_res) 
        | (Some(Val(IntLit(r))), 
           Some(Val(IntLit(g))), 
           Some(Val(IntLit(b))), 
           Some(Val(IntLit(a)))) -> 
             Some((r, g, b, a))
        | _ -> 
          (* indeterminate color shown as X *)
          None 
      in 
      
      (* generate splice editors *)
      let size = FixedWidth(20) in 
      r_editor <- editor(model.r, size)
      g_editor <- editor(model.g, size)
      b_editor <- editor(model.b, size)
      a_editor <- editor(model.a, size)
      
      (* ... now we can render the UI ... *)
    
  let update : 
      Model -> Action -> UpdateCmd(Model) =
    fun model (ClickOn c) -> do 
      set_splice(model.r, IntLit(c.r))
      set_splice(model.g, IntLit(c.g))
      set_splice(model.b, IntLit(c.b))
      set_splice(model.a, IntLit(c.a))
      return model
  
  let expand : Model -> (Exp, List(SpliceRef)) = 
    fun model -> (`fun r g b a -> (r, g, b, a)`, 
      [model.r, model.g, model.b, model.a])
}
\end{lstlisting}
\vspace{-8px}
\caption{Livelit Implementation}
\label{fig:color-impl}
\end{figure}

\noindent
We now take the perspective of a livelit provider.
Fig.~\ref{fig:color-impl}
defines \li{\$color} from Fig.~\ref{fig:color}. We omit certain 
incidental details and use unimplemented syntactic sugar, including
Haskell-style \li{do} notation and quasiquotation, for presentation.

Livelit definitions are scoped and packaged like 
any other definition. They consist of a declaration and an implementation. 
Line 2 of Fig.~\ref{fig:color-impl} is \li{\$color}'s declaration,
which defines its name and its {expansion type}, \li{Color}, defined on Line 1. 
Livelit parameters would also appear here as shown in the declarations in Sec.~\ref{sec:expansion-typing}. 
The declaration is part of the client interface, 
as discussed in Sec.~\ref{sec:expansion-typing}-\ref{sec:parameterization}. 

\subsection{Livelit Implementations}
The curly braces delimit the livelit's implementation, which is not intended to be seen 
by clients. 
Every livelit implementation must include all of the definitions shown in Fig.~\ref{fig:color-impl}, 
which we will discuss in turn below. 
Livelits are implemented using a variation on the functional model-view-update
architecture popularized by Elm \cite{ElmArchitecture}. We add a fourth component,
expansion generation. In addition, we use a simple monadic framework (\emph{a la} Haskell \cite{marlow2010haskell}) to provide the necessary  
interface between the livelit and the editor, while retaining
a pure functional programming model (monadic commands, like \li{new_splice} discussed below, are executed by the editor).

\subsubsection{Model}\label{sec:model}
The state of a livelit's GUI is determined by its model value. 
Line 3 of Fig.~\ref{fig:color-impl} specifies the corresponding \emph{model type},
here a labeled 4-tuple of \emph{splice references}, one for each of the four splices
that appear in the GUI in Fig.~\ref{fig:color}.
The model is how the GUI state is persisted in the syntax tree, so 
 the system requires that the model type supports automatic serialization
 (so no functions can appear in models).

The \li{init} value on Line 8 determines the value of the model
when the livelit is first invoked in the editor.
It is a command in the \li{UpdateCmd} monad, further discussed in Sec.~\ref{sec:def-update}, that returns the 
initial model value after generating four new splices 
using the \li{new_splice} command:
\begin{lstlisting}[numbers=none]
  new_splice : (Typ, Maybe(Exp)) 
               -> UpdateCmd(SpliceRef)
\end{lstlisting}
This command creates a splice
of the given type and, optionally, its initial contents. 
It returns a splice reference, which uniquely identifies that splice.
In this section, bolded types  
are defined in the standard library. The \li{Typ} and \li{Exp} types 
encode the syntax of Hazel's types and expressions 
and we use quasiquotation, e.g. \li{`0`}, as the introduction forms \cite{bawden1999quasiquotation}. 
% If no initial expression is provided, the splice starts as  
% an empty hole.

To ensure \emph{context independence}, 
the system checks that the splice type and initial content are valid 
assuming only the parameters and explicitly specified \li{context} on Line 6.
Here, the context is empty because \li{Int} is a built in type
and the initial expressions are integer literals. 
We use an explicit context, rather than implicitly capturing all bindings 
at the definition site, 
to ensure that private bindings are not unintentionally  
leaked to clients \cite{TLMs}.
% Formally, captured variables can be understood as parameters that have been 
% immediately partially applied.

\subsubsection{Action}
Line 15 defines the \li{Action} type for the \li{\$color} livelit, which 
specifies a single user-initiated action: clicking on a color using
the right half of Fig.~\ref{fig:color}. Actions are emitted
from event handlers (e.g. click handlers) defined in the computed \li{view}, Sec.~\ref{sec:def-view}, 
and actions are consumed by the \li{update} function, Sec. \ref{sec:def-update}, causing a change to the model.

\subsubsection{View}\label{sec:def-view}
The \li{view} function computes the view given the model and access to the commands in 
the \li{ViewCmd} monad (which is distinct from the \li{UpdateCmd} monad). The computed view is a value of type \li{Html(Action)}.  
This type  provides a simple immutable
encoding of an HTML element, where the type parameter is the type of actions that 
are emitted by event handlers that can be attached to elements, e.g.
\li{on_click} and so on. 
We elide the details of the particular user interface in Fig.~\ref{fig:color}, 
but note that livelit implementations can themselves invoke other  
livelits, e.g. the \li{view} function could use livelits for 
expressing user interface widgets and layouts (not shown). 
Instead, we focus on three mechanisms exposed by \li{ViewCmd}: live evaluation,
splice editors, and result rendering.


\paragraph{Live Evaluation}
\label{sec:live-evaluation-def}
As discussed in Sec.~\ref{sec:live-evaluation}, 
the view can depend on the result of evaluating a splice or a parameter
under the closure the client has selected.
(Parameters, not shown in this example but discussed in Sec.~\ref{sec:parameterization}, operate like splices and have type \li{SpliceRef}
within the livelit definition.)
The interface between the view and the live evaluator 
is via the 
following command:
% Actually, it can depend on an encoded function of 
% multiple splices as long as the function is well-typed using only the capture set, 
% under the closure that the client has selected using one of the following
% commands:
\begin{lstlisting}[numbers=none,xleftmargin=0cm]
eval_splice : SpliceRef -> ViewCmd(Maybe(Result))
\end{lstlisting}

%eval_fun : (Exp, List(SpliceRef)) -> ViewCmd(Maybe(Result))

The \li{None} case 
arises when evaluation is not possible, e.g. because no closures are collected 
or because no value has been collected for a variable used in the splice. 
In practice, there is an implicit hole at the end of each cell in Hazel, so 
livelits will have at least one closure containing results for at least 
the top-level bindings. Function argument values may not be available 
if that function has not been applied.
% because the free variables in the selected closure overlap with the free 
% variables in the provided expression. This occurs only when the user has selected
% a closure that appears under a binder in the evaluation result (e.g. because 
% the result is a lambda). 
% We discuss this further in Sec.~\ref{sec:calculus-closure-collection}.

If available, \li{Result} distinguishes two possibilities:
\begin{lstlisting}[numbers=none]
type Result = Val(Exp) | Indet(Exp)
\end{lstlisting}
The \li{Val} case arises when evaluation produces a value, whereas the 
\li{Indet} case arises when evaluation results in an indeterminate expression,
i.e. an expression that cannot be fully evaluated due to holes 
in critical positions \cite{HazelnutLive}.

Lines 26-34 determine a color to display in the color preview 
if all four splices evaluate to integers. Otherwise, there is not 
enough information to determine a color. The livelit 
indicates this situation by disabling the color preview.

Livelits can attempt to offer feedback even when the result is indeterminate,
because indeterminate expressions might nevertheless contain useful information.
For example, a livelit that previews a sequence of notes as audio might be able 
to handle a list of notes where certain notes are missing, i.e. holes, by 
playing silence.
This behavior is highly domain-specific, so each livelit provider must decide 
whether and how indeterminate results are supported.

\paragraph{Splice Editors}
The view includes an editing area for each splice. These editors must support all of 
Hazel's editing services. To support this, the \li{view} function
can request an editor with a given dimension (in character units) for a given splice:
\begin{lstlisting}[numbers=none]
editor : (SpliceRef, Dim) -> ViewCmd(Html(a))
\end{lstlisting}
The result is an opaque \li{Html(a)} value that the remainder of the function 
can place where needed. When the livelit is rendered, this part of the tree is 
under the control of Hazel. The \li{Dim} parameter currently supports only a fixed
character width, with overflow causing scrolling, but in the future we plan to offer 
to offer more flexible layout options.

\paragraph{Result Rendering}
Some livelit views needs to include 
a rendered evaluation result. For example, each of the cells in the \li{\$dataframe}
livelit in Fig.~\ref{fig:grading} show the evaluation result for the corresponding 
cell. Only the formula bar at the top is an editor. To support this, the \li{view}
function can use the \li{result_view} command, which mirrors the \li{editor} command:
\begin{lstlisting}[numbers=none]
result_view : (SpliceRef, Dim) 
              -> ViewCmd(Maybe(Html(a)))
\end{lstlisting}
\subsubsection{Update}\label{sec:def-update}
When the user triggers an event in a livelit view, it emits an \li{Action}.
The system responds by calling the \li{update} function to determine how 
this action should affect the model and, in some cases, the splices. 

% We discussed in Sec.~\ref{sec:model} that the \li{UpdateCmd}
% allows new splices to be generated in response to user actions. 
% For example, the \li{\$dataframe} livelit uses this when new rows
% or columns are added.

In Fig.~\ref{fig:color-impl}, we see that the \li{\$color} livelit responds to 
the \li{ClickOnColor} action by invoking the \li{set_splice} command to overwrite 
the current splices with integer literals determined based on which color the user
clicked on:
\begin{lstlisting}[numbers=none]
  set_splice : (SpliceRef, Exp) -> UpdateCmd ()
\end{lstlisting}
As with \li{new_splice} described in Sec.~\ref{sec:model},
the system maintains context independence by checking the expression against the splice type 
and only allows use of the specified \li{context}.

When the model is updated, a new view is 
computed. The system then performs a diff between the old and new view in order to 
efficiently perform the necessary imperative updates to the editor's visual state.
Changes to splices can also cause the view to be recomputed, because the view might 
evaluate the splices. The \li{UpdateCmd} monad does not itself 
have the ability to request evaluation (\li{eval_splice}), because the model should not depend directly  
on which closure the user has selected. 
Of course, the \li{view} might emit 
result-dependent actions when appropriate.

\subsubsection{Expansion}
\label{sec:expansion}
The ultimate purpose of a livelit is to fill the hole where it appears by generating an expansion,
i.e. an expression of the expansion type, here \li{Color}.
The \li{expand} function determines this expansion based on the model 
(which encodes the GUI state, and so usually contains information, like splice references
and widget states, 
that should not persist into run-time as part of the expansion).

The expansion can include the spliced expressions, 
but the system does not make the expression  
available directly as a value of type \li{Exp}. 
Instead, \li{expand} returns a \emph{proto-expansion}, of type \li{Exp}, 
paired with a list of \li{SpliceRef}s (which can come from 
both splices and parameters as discussed above).
The {proto-expansion}
 is an encoding (i.e. a value of type \li{Exp}) of a function 
 (here curried)
 that takes an argument for each listed \li{SpliceRef}. 
 That argument is of the corresponding splice 
type, which was provided when the splice was initialized. 
 The return type of the proto-expansion is the expansion type.
 So here, the proto-expansion for \li{\$color} must be a function of type 
 \li{Int -> Int -> Int -> Int -> Color}.

Note that Hazel does \emph{not} statically check the definition of \li{expand}
to ensure that the encoded proto-expansion has this type. Instead, the proto-expansion 
is \emph{validated} at each livelit invocation site, with errors reported to the client. 
A \emph{typed quotation} system as in, e.g., MetaOCaml \cite{DBLP:conf/flops/Kiselyov14}, could be adapted to allow for definition-site verification 
in lieu of client-site validation, as discussed in \cite{TLMs}, but we leave the integration 
of such a system into Hazel as future work. This would be non-trivial:  
the type of the quotation depends on the type of each splice in 
the splice list, which may be of model-dependent length (e.g. in the case of \li{\$dataframe}).

This parameterization strategy makes it simple to enforce the binding discipline described in Sec.~\ref{sec:hygiene}. 
The proto-expansion 
can depend only on the \li{context}, whereas the splices are entered by the client and so they can 
depend only on the client site typing context. The use of function application ensures
that splices are capture avoiding as discussed in Sec.~\ref{sec:hygiene}. 
We consider this more formally next.

%\section{A Simply Typed Livelit Calculus}\label{sec:livelit-calculus}

\begin{figure}
    \vspace{-3px}
    \[
    \arraycolsep=3pt\begin{array}{rlcl}
        \mathsf{Typ} & \tau & ::= &
                                    % \tnum ~\vert~
                                    \tarr{\tau_1}{\tau_2} ~\vert~
                                    \tprod{\tau_1}{\tau_2} ~\vert~
                                    \tunit ~\vert~
                                    \tsum{\tau_1}{\tau_2} ~\vert~
                                    t ~\vert~
                                    \trec{t}{\tau}\\
        \mathsf{UExp} & \hat{e} & ::= & 
                                 x ~\vert~
                                 \hlam{x}{\hat e} ~\vert~
                                 \hap{\hat e_1}{\hat e_2} ~\vert~
                                 ... ~\vert~
                                %  \hpair{\hat e_1}{\hat e_2} ~\vert~
                                %  \hprl{\hat e} ~\vert~
                                %  \hprr{\hat e} ~\vert~
                                %  \htriv ~\vert~
                                %  \hinL{\hat e} ~\vert~
                                %  \hinR{\hat e} ~\vert~
                                %  \hroll{\hat e} ~\vert~
                                %  \hunroll{\hat e} \\
                                 \hehole{u} ~\vert~
                                 \haplivelit{u}{a}{\dtxt{model}}{\splices}\\
        \mathsf{EExp} & e & ::= & x ~\vert~ \hlam{x}{e} ~\vert~ \hap{e_1}{e_2} ~\vert~ ... ~\vert~ \hehole{u}\\
        \mathsf{IExp} & d & ::= & x ~\vert~ \hlam{x}{d} ~\vert~ \hap{d_1}{d_2} ~\vert~ ... ~\vert~ \dehole{u}{\sigma}\\
        \mathsf{Splice} & \psi & ::= & \hsplice{\hat e}{\tau}
    \end{array}\]
    \caption{Syntax of types, $\tau$, unexpanded expressions, $\hat{e}$, expanded expressions, $e$, and internal expressions, $d$.
    Here, $x$ ranges over variables, $u$ over hole names, and $\livelitname{a}$ over livelit names.
    We write $\splat{\psi_i}$ for a finite sequence of $n \geq 0$ splices,
    and $\sigma$ for finite substitutions of $n \geq 0$ internal expressions for variables, $[d_1/x_1, \cdots, d_n/x_n]$.
    We elide standard forms
    related to product, sum, and recursive types.
    }
    \label{fig:syntax}
    \end{figure}

In order to communicate the semantics of livelits
independently
of the specifics of the Hazel environment and the web platform, 
we now specify a simply typed \emph{livelit calculus}.

Fig.~\ref{fig:syntax} specifies the syntax of the livelit calculus.
Programs are written as \emph{unexpanded expressions}, $\hat e$, which are \emph{expanded} to
\emph{external expressions}, $e$, before being \emph{elaborated}
to \emph{internal expressions}, $d$, for evaluation. All three sorts are classified
by the same types, $\tau$. We include partial functions, products, sums, and recursive
types, all in their standard form \cite{pfpl}, but this specific type structure is not critical.
Any language expressive
enough to encode its own abstract syntax 
would be a suitable basis for a livelit calculus. 

We begin in Sec.~\ref{sec:external-and-internal-lang} with a terse
overview of the external and internal languages,
which are adapted from Hazelnut Live \cite{HazelnutLive}.
We then detail livelit expansion (Sec.~\ref{sec:calculus-expansion}) and 
liveness via closure collection (Sec.~\ref{sec:calculus-closure-collection}). 
% and parameterization (Sec.~\ref{sec:calculus-parameterization}).


\subsection{Background: External and Internal Language}\label{sec:external-and-internal-lang}
The external and internal languages are straightforward adaptations of the
external and internal languages of \emph{Hazelnut Live},
a typed lambda calculus that assigns static and dynamic meaning to programs with holes,
notated $\hehole{u}$ where $u$ is a \emph{hole name} \cite{HazelnutLive}.
We omit non-empty holes (which internalize type inconsistencies \cite{Hazelnut}) and type holes
(which operate like the unknown type from gradual type theory \cite{Siek06a,Hazelnut}).
These mechanisms are orthogonal to livelits and are included in our implementation.

External expressions, e, are governed by a typing judgement of the form, $\hasType{\Gamma}{e}{\tau}$,
where the typing context, $\Gamma$,
is a finite set of typing assumptions of the form $x : \tau$ \cite{pfpl}.

The internal language is a contextual type theory \cite{Nanevski2008}, i.e. the typing judgement is
of the form $\hasTypeD{\Delta}{\Gamma}{d}{\tau}$ where $\Delta$ is a finite set of hole typing
assumptions of the form $u :: \tau[\Gamma]$.
We need this hole typing context only for the internal language because, although hole names are assumed
unique in the external language, they can be duplicated during evaluation of internal expressions.
The hole context ensure that each closure associated with hole $u$, written $\dehole{u}{\sigma}$,
shares a type and can be filled
with expressions valid under a shared context.

\newcommand{\tnat}{\mathsf{nat}}

External expressions are given dynamic meaning by typed elaboration to internal expressions, $d$,
according to the judgement $\elabs{\Gamma}{e}{d}{\tau}{\Delta}$.
The main purpose of elaboration is to initialize the substitution $\sigma$ on each hole closure,
which operates to capture
the substitutions that have occurred around that hole during evaluation. The key rule is:
\begin{mathpar}
\inferrule[Elab-Hole]{ }{
    \elabs{\Gamma}{\hehole{u}}{\dehole{u}{\idof{\Gamma}}}{\tau}{u :: \tau[\Gamma]}
}
\end{mathpar}
The substitution is initially the identity substitution, $\idof{\Gamma}$, i.e. the
substitution that maps each variable in $\Gamma$ to itself, because no substitutions have yet occurred. For example,
\[ \elabs{ }{(\hlam{x}{\hehole{u}})~\hnum{5}}{(\hlam{x}{\dehole{u}{[x/x]}})~{\hnum{5}}}{\tnat}{u :: \tnum[x : \tnat]} \]


During evaluation, $\evalsTo{d}{d'}$, the closure's substitution accumulates the substitutions that occur. For example,
the internal expression above evaluates as follows:
\[
  \evalsTo{(\hlam{x}{\dehole{u}{[x/x]}})~{\hnum{5}}}{
      \dehole{u}{[\hnum{5}/x]}
  }
\]
% We will use hole closures to support live programming with livelits in Sec.~\ref{sec:calculus-closure-collection}.


All of these judgements are adapted directly from Hazelnut Live,
differing only in that we use a declarative rather than an algorithmic (bidirectional)
formulation for simplicity. Rather than restating the rules, we simply state the key governing
metatheorems and refer the reader to the prior work and our Agda mechanization (Sec.~\ref{sec:agda}) for the full details \cite{HazelnutLive}.

First, we have that elaboration preserves typing.
\begin{theorem}[Typed Elaboration]
    If $\hasType{\Gamma}{e}{\tau}$ then $\elabs{\Gamma}{e}{d}{\tau}{\Delta}$ for some $d$ and $\Delta$ such
    that $\hasTypeD{\Delta}{\Gamma}{d}{\tau}$.
\end{theorem}

Next, we have that evaluation of a closed expression with holes produces a final (i.e. irreducible) preserves typing.
\begin{theorem}[Preservation]
    If $\hasTypeD{\Delta}{\cdot}{d}{\tau}$ and $\evalsTo{d}{d'}$ then $\isFinal{d'}$ and $\hasTypeD{\Delta}{\cdot}{d'}{\tau}$.
\end{theorem}

\subsection{Expansion}\label{sec:calculus-expansion}

The novelty of the livelit calculus is entirely in its handling of unexpanded expressions,
$\hat e$, which are given meaning by typed expansion to external expressions,
$e$, according to the judgement $\expands{\Phi}{\Gamma}{\hat e}{e}{\tau}$
defined in Fig.~\ref{fig:expansion}.
Unexpanded expressions mirror external expressions but for the presence of livelit invocations.
The rules for the mirrored 
forms like variables and functions, both shown in Fig.~\ref{fig:expansion},
are simple.

\subsubsection{Livelit Contexts}\label{sec:livelit-contexts}

Livelit definitions are collected in the livelit context, $\Phi$, which
maps livelit names to livelit definitions of the following form:
\[ \livelitCtxEntry{a}{\taut{expand}}{\taut{model}}{\dtxt{expand}} \]
Here, $\taut{expand}$ is the expansion type, $\taut{model}$ is the model type,
and $\dtxt{expand}$ is the expansion function, which generates an expansion given a model.
We omit the logic related to view computations and actions, which are tied to a particular 
UI framework and have only indirect semantic significance.

% For a livelit context to be well-formed, we require that each livelit definition be well-formed.
% This in turn requires that the expansion function be hole-free and have the correct type.
\begin{definition}[Livelit Context Well-Formedness]
    A livelit context $\Phi$ is well-formed if and only if for each livelit definition,
    $\livelitCtxEntry{a}{\taut{expand}}{\taut{model}}{\dtxt{expand}} \in \Phi$,  we have
    $\hasType{}{\dtxt{expand}}{\tarr{\taut{model}}{\expType}}$.
\end{definition}
Here, $\expType$ stands for a type whose values isomorphically encode
external expressions. The isomorphism is mediated in one direction by
the encoding judgement $\encodeExp{e}{d}$ and the other by the decoding judgement $\decodeExp{d}{e}$.
Any scheme is sufficient, so we leave it as a matter of implementation.
The simplest approach is to define $\expType$ as a recursive sum type,
with one arm for each form of external expression (cf. \cite{TSLs}).

For simplicity, we assume that the livelit context is provided \emph{a priori} and therefore
that the expansion function is already closed and fully elaborated.
In practice, the livelit context would be controlled by a definition form in the language
that allows the definition to itself invoke livelits.
This would require a staging mechanism,
because we need to evaluate expansion functions in prior definitions to be able to
expand subsequent definitions.
There are a number of ways to support the necessary staging in practice, e.g.
via explicit staging primitives \cite{DBLP:conf/icfp/Flatt02},
by requiring that these definitions appear in separately compiled packages \cite{TLMs},
or by using live
programming mechanisms such as those in Hazel
to evaluate ``up to'' each definition before proceeding \cite{HazelnutLive}.
% This choice is orthogonal to the mechanisms of interest in this section.

\subsubsection{Hygienic Livelit Expansion}
\begin{figure}
    \begin{mathpar}
        \inferrule[EVar]{
            x : \tau \in \Gamma
        }{
            \expands{\Phi}{\Gamma}{x}{x}{\tau}
        }
        
        \inferrule[ELam]{
            \expands{\Phi}{\Gamma, x : \taut{in}}{\hat e}{e}{\taut{out}}
        }{
            \expands{\Phi}{\Gamma}{\hlam{x}{\hat e}}{\hlam{x}{e}}{\tarr{\taut{in}}{\taut{out}}}
        }~~~~~~~~\cdots

        % \inferrule[EAp]{
        %     \expands{\Phi}{\Gamma}{\hat e_1}{e_1}{\tarr{\taut{in}}{\taut{out}}}\\\\
        %     \expands{\Phi}{\Gamma}{\hat e_2}{e_2}{\taut{in}}
        % }{
        %     \expands{\Phi}{\Gamma}{\hap{\hat e_1}{\hat e_2}}{\hap{e_1}{e_2}}{\taut{out}}
        % }
        \inferrule[EApLivelit]{
            \livelitCtxEntry{a}{\taut{expand}}{\taut{model}}{\dtxt{expand}} \in \Phi\\\\
            \hasType{ }{\dtxt{model}}{\taut{model}}\\
            \evalsTo{\hap{\dtxt{expand}}{\dtxt{model}}}{\dtxt{encoded}}\\
            \decodeExp{\dtxt{encoded}}{\etxt{pexpansion}}\\\\
            \hasType{ }{\etxt{pexpansion}}{\tarr{\{\tau_i\}_{i < n}}{\taut{expand}}}\\
            \{ \expands{\Phi}{\Gamma}{\hat e_i}{e_i}{\tau_i} \}_{i < n}
        }{
            \expands{\Phi}{\Gamma}{\haplivelit{u}{a}{\dtxt{model}}{\splat{\hsplice{\hat e_i}{\tau_i}}}}{
                \hap{\etxt{pexpansion}}{\{e_i\}_{i < n}}
            }{\taut{expand}}
        }
    \end{mathpar}
    \caption{Expansion}
    \label{fig:expansion}
    \end{figure}
Unexpanded expressions are unique in that they include livelit invocations:
 \[\haplivelit{u}{a}{\dtxt{model}}{\splices}\]
Here, $\livelitname{a}$ names the livelit
 being applied. Livelits can be understood as filling holes, so $u$ identifies the hole
 that is, conceptually, being filled.
The current state of the livelit is determined by the current model value, $\dtxt{model}$,
together with the splice list, $\splices$. Each splice $\psi_i$
is of the form $\hsplice{\hat e_i}{\tau_i}$, where $\hat e_i$ is the spliced expression
itself (unexpanded, so it may contain other livelits) and $\tau_i$ is the type of that splice,
as determined when the livelit definition first requested the splice (discussed in Sec.~\ref{sec:model}).

Rule \rulename{EApLivelit} performs livelit expansion. Its premises, in order, operate as follows:
\begin{enumerate}[itemsep=3px,leftmargin=*]
    \item \textbf{Lookup.} The first premise looks up the livelit name in the livelit context.
    \item \textbf{Model Validation.} The second premise serves to ensure that the model value, $\dtxt{model}$, is of the
    specified model type, $\taut{model}$.
    \item \textbf{Expansion.} The third premise applies the expansion function, $\dtxt{expand}$, to the model value, $\dtxt{model}$,
    producing the encoded parameterized expansion, $\dtxt{encoded}$, which, by the definitions given above, is of type $\mathsf{Exp}$.
    \item \textbf{Decoding.} The fourth premise decodes $\dtxt{encoded}$, producing the \emph{parameterized expansion}, $\etxt{pexpansion}$.
    \item \textbf{Expansion Validation.} The fifth premise checks that {parameterized expansion} is a function that returns a value of the expansion type, $\taut{expand}$, when applied (in curried fashion, though this is not critical)
    to the splices, whose types, $\splat{\tau_i}$, are explicitly recorded in the livelit.
    We must take care to ensure that the parameterized expansion is \emph{context independent},
    i.e. that it cannot depend on the particular bindings available in the call site typing context, $\Gamma$.
    % Context independence is one facet of what is colloquially known as \emph{hygiene} \cite{TLMs}.
    In this simple formulation of the system, we maintain context independence by
    requiring that the parameterized expansion be
    closed. Consequently, any necessary helper functions used in the expansion must be provided explicitly by the client
    via a splice. 
    In Sec.~\ref{sec:livelit-definitions}, we discussed how the use of explicit capture sets eliminates this client burden.
    For simplicity, we omit capture sets from the calculus, but see \cite{TLMs} for a formal treatment.
    \item \textbf{Splice Expansion.} The sixth premise inductively expands each of the spliced expressions in the same context as the livelit
    invocation itself.
\end{enumerate}
The conclusion of the rule then applies the parameterized expansion to the expanded splices.
By applying the splices as arguments, we maintain \emph{capture avoidance} -- splices cannot capture variables
bound internally to the expansion because beta reduction performs capture-avoiding substitution. 
% Capture avoidance is the other facet of \emph{hygiene} \cite{TLMs}.

The typed expansion process is governed by the following metatheorem, which establishes that the expansion
is indeed an external expression of the indicated type.



\begin{theorem}[Typed Expansion]
    If $\expands{\Phi}{\Gamma}{\hat e}{e}{\tau}$ then $\hasType{\Gamma}{e}{\tau}$.
\end{theorem}
% \begin{proof}
%     We proceed by rule induction on the assumption.
%     The cases involving the standard forms follow by straightforward induction.
%     The only interesting case is the \rulename{EApLivelit} case.
%     In this case, we have that $e = \hap{\etxt{pexpansion}}{\splat{e_i}}$.
%     By assumption, we have $\{ \expands{\Phi}{\Gamma}{\hat e_i}{e_i}{\tau_i} \}_{i < n}$.
%     By induction, we therefore have $\{ \hasType{\Gamma}{e_i}{\tau_i} \}_{i < n}$.
%     In addition, by assumption we have that $\hasType{}{\etxt{pexpansion}}{\tarr{\splat{\tau_i}}{\taut{expand}}}$.
%     By weakening, we therefore have $\hasType{\Gamma}{\etxt{pexpansion}}{\tarr{\splat{\tau_i}}{\taut{expand}}}$.
%     Finally, by iterated application of the typing rule for function application over the splices,
%     using the judgements just established,
%     we have $\hasType{\Gamma}{\hap{\etxt{pexpansion}}{\splat{e_i}}}{\taut{expand}}$ as desired.
% \end{proof}

When composed with the Typed Elaboration theorem and the Type Safety properties established for the internal
language, we achieve end-to-end type safety: every well-typed unexpanded
expression expands to a well-typed external expression, which in turn elaborates to a well-typed internal
expression, which in turn evaluates in a type safe manner.

\subsubsection{Agda Mechanization}
\label{sec:agda}
We have mechanized the external and internal language and 
typed expansion and proven the aforementioned theorems
using the Agda proof assistant, building on the Agda mechanization 
of Hazelnut Live \cite{HazelnutLive}. This mechanization is available in the supplement. 


\subsection{Live Feedback via Closure Collection}\label{sec:calculus-closure-collection}
In order to support live feedback, a livelit needs to be able to ask the system
to evaluate expressions under one of the closures associated with the livelit.
This mechanism was introduced by example in Sec.~\ref{sec:closure-collection-example}.
In this section, we will formalize the process of efficiently collecting closures.

\subsubsection{Proto-Environment Collection}
We begin by generating an alternative expansion,
called the \emph{cc-expansion},
where each livelit invocation expands to an empty hole applied to its splices. In other words,
the hole is in place of the parameterized expansion, which we instead record in a \emph{cc-context}, $\Omega$, on the side.
The key rule for the cc-expansion judgement, $\ccexpands{\Phi}{\Gamma}{\hat e}{e}{\tau}{\Omega}$, is:
\begin{mathpar}
\inferrule[CCApLivelit]{
    \{ \ccexpands{\Phi}{\Gamma}{\hat e_i}{e_i}{\tau_i}{\Omega_i} \}_{i < n}\\
    \expands{\Phi}{\Gamma}{\haplivelit{u}{a}{\dtxt{model}}{\splat{\hsplice{\hat e_i}{\tau_i}}}}{
        \hap{\etxt{pexpansion}}{\splat{e_i}}
    }{\taut{expand}}\\
    \elabs{ }{\etxt{pexpansion}}{\dtxt{pexpansion}}{\tarr{\splat{\tau_i}}{\taut{expand}}}{\Delta}\\
    \Omega = \cup_{i<n} \Omega_i \cup \{\Omegaitem{u}{\dtxt{pexpansion}}\}
}{
    \ccexpands{\Phi}{\Gamma}{\haplivelit{u}{a}{\dtxt{model}}{\splat{\hsplice{\hat e_i}{\tau_i}}}}{\hap{\hehole{u}}{\splat{e_i}}}{\taut{expand}}
    {\Omega}
}
\end{mathpar}

We then elaborate and evaluate the cc-expansion. 
The result will contain some number of hole closures
for each livelit hole. We call these the proto-closures and their environments the proto-environments for that livelit hole.
\begin{definition}[Proto-Closure Collection]
If $\ccexpands{\Phi}{\cdot}{\hat e}{e}{\tau}{\Omega}$ and $\elabs{}{e}{d}{\tau}{\Delta}$
and $\evalsTo{d}{d'}$ and $u \in \domof{\Omega}$ then $\protoEnvsOf{\Phi}{\hat e}{u} = \{ \sigma \mid \dehole{u}{\sigma} \in d' \}$.
    % protoclosures(\hat e, u) = { sigma | hole^u_sigma in d' } where \hat e cc-expands to e and e elaborates to d and d evaluates to d'
\end{definition}

\subsubsection{Closure Resumption}
A proto-environment for a livelit hole might itself contain a proto-closure for another livelit hole,
which is problematic for the reasons detailed in Sec.~\ref{sec:closure-collection-example}.
% For example, in Fig.~\ref{fig:color}, the proto-environment for the \li{\$color} livelit contained the
% proto-closure for the \li{\$slider} livelit, via the \li{baseline} variable.
% If we use proto-environments for live evaluation, then the actual value of that variable
% would not be available (here, the color could not be displayed).
Consequently, the second step of closure collection, called \emph{closure resumption}, is to fill any livelit holes that appear in the proto-environments for other livelit holes.
We do so by filling them using the parameterized expansions gathered in $\Omega$ and then resuming
evaluation where appropriate.
Formally, this involves the hole filling operation $\instantiate{d_1}{u}{d_2}$ for Hazelnut Live
(which derives from the metavariable instantation operation of contextual modal type theory \cite{HazelnutLive,Nanevski2008}).
This operation
fills every closure for hole $u$ in $d_2$ with $d_1$.
Unlike substitution, hole filling is
not capture-avoiding. Instead, the environment on each of these closures is applied to $d_1$
as a substitution, i.e. the delayed substitutions captured in the environment are realized.
In this case, however, the parameterized expansion is necessarily closed due to
the context independence discipline we maintain in Rule \rulename{EApLivelit},
so hole filling amounts to syntactic replacement.

Formally, we begin by defining an operation $\fillof{\Omega}{\sigma}$ which acts on proto-environments
to fill the livelit holes.

\begin{definition}[Livelit Hole Filling] ~
    \begin{enumerate}
        \item $\fillof{\Omega}{[d_1/x_1, \ldots, d_n/x_n]} = $\\
        $\hspace{10px}[\fillof{\Omega}{d_1}/x_1, \ldots, \fillof{\Omega}{d_n}/x_n]$
        \item $\fillof{\Omega}{d} = \instantiateB{\dtxt{pexpansion}}{u}_{\Omegaitem{u}{\dtxt{pexpansion}} \in \Omega}{d}$
    \end{enumerate}
\end{definition}

This first step may cause certain expressions to become non-final, because the filled hole is no longer
blocking evaluation. We therefore define an operation $\resumeof{\sigma}$ that resumes evalution for all closed expressions in $\sigma$.
(The only open expressions that might remain are the initial variables from the identity substitution generated by elaboration. Some closures appear under binders in the final result, so these variables will not have yet recorded a substitution.)
\begin{definition}[Environment Resumption] ~
    \begin{enumerate}
        \item $\resumeof{[d_1/x_1, \ldots, d_n/x_n]} = $\\
        $[\resumeof{d_1}/x_1, \ldots, \resumeof{d_n}/x_n]$
        \item $\resumeof{d} = d'$ if $\fvof{d} = \emptyset$ and $\evalsTo{d}{d'}$
        \item $\resumeof{d} = d$ if $\fvof{d} \neq \emptyset$
    \end{enumerate}
\end{definition}

Finally, we can produce the final set of environments by filling and resuming the proto-environments.
\begin{definition}[Environment Collection]
    If $\ccexpands{\Phi}{\cdot}{\hat e}{e}{\tau}{\Omega}$
    % and $\elabs{}{e}{d}{\tau}{\Delta}$
    % and $\evalsTo{d}{d'}$ and $u \in \domof{\Omega}$
    then \[\envsOf{\Phi}{\hat e}{u} = \{\resumeof{\fillof{\Omega}{\sigma}} \mid \sigma \in \protoEnvsOf{\Phi}{\hat e}{u}\}\]
\end{definition}

This same fill and resume operation can be used to avoid recomputation when evaluating the fully expanded version of the user's program.
If the editor has already performed environment collection, then it can simply continue from where it left off
by filling and resuming
the remaining top-level livelit holes (those that do not appear in a proto-environment).

The correctness of the mechanisms described in this section rest on the fact that evaluation commutes with hole filling
in the pure setting.
% This only holds when the language is pure: evaluation order does not matter
% so the system is free to evaluate around holes initially,
% and then return to finish the job once they have been filled.

\begin{theorem}[Post-Collection Resumption]
    If $\ccexpands{\Phi}{\cdot}{\hat e}{\etxt{cc}}{\tau}{\Omega}$ and $\elabs{}{\etxt{cc}}{\dtxt{cc}}{\tau}{\Delta}$
    and $\evalsTo{\dtxt{cc}}{\dtxt{cc}'}$ and $\resumeof{\fillof{\Omega}{\dtxt{cc}'}} = d_1$
    and $\expands{\Phi}{\cdot}{\hat e}{\etxt{full}}{\tau}$
    and $\elabs{}{\etxt{full}}{\dtxt{full}}{\tau}{\Delta}$
    and $\evalsTo{\dtxt{full}}{d_2}$ then $d_1 = d_2$.
\end{theorem}
\begin{proof}
    The key observation is that filling the livelit holes in the cc-expansion gives the full expansion,
    i.e. $\fillof{\Omega}{\dtxt{cc}} = \dtxt{full}$. Resumption is simply evaluation for closed expressions.
    By commutativity of hole filling, established in the prior work \cite{HazelnutLive},
    we can delay hole filling until $\dtxt{cc}$ has first been evaluated to
    $\dtxt{cc}'$.
\end{proof}

In a language with side effects, one could evaluate 
cc-expansions in an alternative mode where the full expansion of each livelit invocation is evaluated 
in the order that corresponding livelit hole, and any expressions blocked on it are resumed immediately. 
The results would need to be recorded in memory for use in the filling phase. This would ensure that
side effects happen in the same order and only once. %A formal specification is left as future work.
Alternatively, an imperative language might use a different mechanism to make
environments available to livelits, e.g. by environment logging \emph{a la} Lamdu \cite{lamdu}.
This is more limited in situations where a livelit is never evaluated, as sometimes 
occurs during development of a livelit in a function which has yet to be applied, 
but does not require cc-expansion. We leave further exploration of this design space 
to future work.

% \subsection{Parameterization}

% Add livelit expressions, livelit abbreviations,
% partial application support. Talk about how to work around the strict context
% independence discipline we have imposed.

% \subsection{Edit Action Semantics}
% So far, we have considered only the semantics of an individual editor state.
% However, livelits are useful because they evolve as the user interacts
% with them. In particular, the livelit's model, $\dtxt{model}$ can evolve
% due to livelit-specific user interactions. Furthermore, a livelit can also
% create and delete splices when a user interaction demands it (unlike functions, which have a fixed
% number of arguments).


%\input{extensions}



%\clearpage
\bibliography{references,all.short,hazel_NSF}

\clearpage
\appendix
% !TEX root = hazelnut-dynamics.tex

% \begin{figure}[p]
% \judgbox
%   {\pexpandSyn{\hGamma}{\pPhi}{\pexp}{\hexp}{\htyp}}
%   {$\pexp$ expands to $\hexp$ which synthesizes type $\htyp$}
% \begin{mathpar}
% \inferrule[SPEConst]{ }{
%   \pexpandSyn{\hGamma}{\pPhi}{c}{c}{b}
% }
% \and
% \inferrule[SPEAsc]{
%   \pexpandAna{\hGamma}{\pPhi}{\pexp}{\hexp}{\htyp}
% }{
%   \pexpandSyn{\hGamma}{\pPhi}{\pexp : \htyp}{\hexp : \htyp}{\htyp}
% }
% \and
% \inferrule[SPEVar]{
%   x : \htyp \in \hGamma
% }{
%   \pexpandSyn{\hGamma}{\pPhi}{x}{x}{\htyp}
% }
% \and
% \inferrule[SPELam]{
%   \pexpandSyn{\hGamma, x : \htyp_1}{\pPhi}{\pexp}{\hexp}{\htyp_2}
% }{
%   \pexpandSyn{\hGamma}{\pPhi}{\halam{x}{\htyp_1}{\pexp}}{\halam{x}{\htyp_1}{\hexp}}{\tarr{\htyp_1}{\htyp_2}}
% }
% \and
% \inferrule[SPEAp]{
%   \pexpandSyn{\hGamma}{\pPhi}{\pexp_1}{\hexp_1}{\htyp_1} \\
%   \arrmatch{\htyp_1}{\tarr{\htyp_2}{\htyp}} \\\\
%   \pexpandAna{\hGamma}{\pPhi}{\pexp_2}{\hexp_2}{\htyp_2}
% }{
%   \pexpandSyn{\hGamma}{\pPhi}{\hap{\pexp_1}{\pexp_2}}{\hap{\hexp_1}{\hexp_2}}{\htyp}
% }
% \and
% \inferrule[SPEHole]{ }{
%   \pexpandSyn{\hGamma}{\pPhi}{\hehole{\mvar}}{\hehole{\mvar}}{\tehole}
% }
% \and
% \inferrule[SPNEHole]{
%   \pexpandSyn{\hGamma}{\pPhi}{\pexp}{\hexp}{\htyp}
% }{
%   \pexpandSyn{\hGamma}{\pPhi}{\hhole{\pexp}{\mvar}}{\hhole{\hexp}{\mvar}}{\tehole}
% }
% \end{mathpar}

% \vsepRule

% \judgbox
%   {\pexpandAna{\hGamma}{\pPhi}{\pexp}{\hexp}{\htyp}}
%   {$\pexp$ expands to $\hexp$ which must analyze against type $\htyp$}
% \begin{mathpar}
% \inferrule[APELam]{
%   \arrmatch{\htyp}{\tarr{\htyp_1}{\htyp_2}} \\
%   \pexpandAna{\hGamma, x : \htyp_1}{\pPhi}{\pexp}{\hexp}{\htyp_2}
% }{
%   \pexpandAna{\hGamma}{\pPhi}{\hlam{x}{\pexp}}{\hlam{x}{\hexp}}{\htyp}
% }
% \and
% \inferrule[APESubsume]{
%   \pexpandSyn{\hGamma}{\pPhi}{\pexp}{\hexp}{\htyp'} \\
%   \tconsistent{\htyp}{\htyp'}
% }{
%   \pexpandAna{\hGamma}{\pPhi}{\pexp}{\hexp}{\htyp}
% }
% \end{mathpar}
% \CaptionLabel{Palette Expansion, remaining rules}{fig:palexpapndx}
% \label{fig:expandSyn}
% \label{fig:expandAna}
% \end{figure}


\end{document}
